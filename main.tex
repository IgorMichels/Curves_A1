\documentclass[11pt,letterpaper,twocolumn]{article}
\usepackage[utf8]{inputenc}
\usepackage[portuguese]{babel}
\usepackage{float}
\usepackage{xcolor}
\usepackage{verbatim}
\usepackage{charter}
\usepackage{amsmath}
\usepackage{appendix}
\usepackage{ragged2e}
\usepackage{array}
\usepackage{etoolbox}
\usepackage{fancyhdr}
\usepackage{booktabs}
\usepackage{arydshln}
\usepackage{caption}
\usepackage{subcaption}
\usepackage{enumitem}
\usepackage[bottom=3 cm,top=2.0cm,left=2.0cm,right=2.0cm]{geometry}
\usepackage{graphicx}
\usepackage{mathtools}
\usepackage{multirow}
\usepackage{pdfpages}
\usepackage{subfiles}
\usepackage[compact]{titlesec}
\usepackage{stfloats}
\usepackage{hyperref}
\usepackage{amsfonts}
\usepackage[export]{adjustbox}
\usepackage{amssymb}
\usepackage{mathrsfs}
\usepackage{amsthm}

\newtheorem{theorem}{Teorema}
\newtheorem{definition}{Definição}
\newtheorem{lemma}{Lema}
\newtheorem{example}{Exemplo}
\newtheorem{proposition}{Proposição}

\setlength{\columnsep}{30pt}

%\titlelabel{\thetitle.\quad}

\pagestyle{fancy}
\fancyhf{}
      
\fancyfoot{}
\fancyfoot[C]{\thepage} % page
\renewcommand{\headrulewidth}{0mm} % headrule width
\renewcommand{\footrulewidth}{0mm} % footrule width

\makeatletter
\patchcmd{\headrule}{\hrule}{\color{black}\hrule}{}{} % headrule
\patchcmd{\footrule}{\hrule}{\color{black}\hrule}{}{} % footrule
\makeatother

\definecolor{blueM}{cmyk}{1.0,0.49,0.0,0.47}
    
\begin{document}
\twocolumn[\begin{@twocolumnfalse}

\begin{minipage}{\textwidth}
\vspace{5mm}
    \Large{\textbf{Teorema Fundamental das Curvas Espaciais}} 
    \vspace{3mm}
    
    \large{\textbf{Cristhian Grundmann, Igor Patrício Michels}} 
    \vspace{2mm}\newline

    \fontsize{0.35cm}{0.5cm}\selectfont \textit{Escola de Matemática Aplicada, EMAp/FGV\newline 
    Rio de Janeiro/RJ, Brasil}
    \vspace{1mm} 
    
    \today
\end{minipage}

\small
\vspace{11pt}
\centerline{\rule{0.95\textwidth}{0.4pt}}

\begin{center}
    \begin{minipage}{0.9\textwidth}
        \noindent \textbf{Resumo:} 
        Esse trabalho tem como objetivo enunciar e demonstrar o Teorema Fundamental das Curvas Espaciais, mostrando que uma curva do $\mathbb{R}^3$ pode ser gerada de forma única através de sua curvatura e torção.
        \vspace{4mm}
        
        \noindent \textbf{Palavras chave:} Sistema de EDO's, Curvas Espaciais, Curvatura, Torção.
    \end{minipage}
\end{center}

\centerline{\rule{0.95\textwidth}{0.4pt}}

\vspace{15pt}
\end{@twocolumnfalse}]

\section{Introdução}
\justify

No decorrer da primeira parte do semestre foram dadas diversas definições, além de terem sido provadas inúmeras relações, proposições, lemas e teoremas. Tendo em vista isso, nos propomos a provar um teorema não visto em sala, mas que é de extrema importância para o assunto abordado, o Teorema Fundamental das Curvas Espaciais. Tal teorema diz que, dado um intervalo $I =(a, b)\in \mathbb{R}$ e as funções $\kappa : I\to \mathbb{R}^*_+$ e $\tau : I\to \mathbb{R}$, as quais chamaremos de curvatura e torção, respectivamente, definem, a menos de movimentos rígidos, uma curva no espaço.

Esse enunciado, embora simples, é de enorme curiosidade, uma vez que o mesmo não parece nada imediato. Dessa forma, como motivação desse trabalho, vamos buscar entender como uma curva pode ser definida, e construída, a partir de sua curvatura e torção. Algumas demonstrações podem ser vistas em \cite{cruz}-\cite{picado}. Nossa abordagem se dará de forma similar a utilizada por \cite{coda}.

\section{Preliminares}
\justify
\label{preliminares}

Primeiramente, é interessante trazer algumas definições de análise que irão nos auxiliar

\begin{definition}(Função Lipschitz-contínua.)
    Seja $M$ um espaço métrico qualquer com a métrica $d$. Uma função $F : M\to M$ é Lipschitz-contínua para todo $x$ e $y$ de $M$ se existe alguma constante $L$ de modo que
    \[d(F(x), F(y))\leq L\cdot d(x, y).\]
    
    O ínfimo das constantes $L$ para os quais vale a desigualdade é chamado de constante de Lipschitz.
\end{definition}

\begin{proposition}
    Se $F$ é uma função Lipschitz-contínua em um intervalo $I$, então $F$ é uma função contínua nesse intervalo.
\end{proposition}

\begin{proof}
    Seja $L$ a constante de Lipschitz de $F$ em $I$. Como $F$ é uma função Lipschitz-contínua, segue da definição que vale
    \[|F(x) - F(y)|\leq L\cdot |x - y|, \forall x, y\in I.\]
    
    Dado $\varepsilon > 0$, considere $\delta = \frac{\varepsilon}{L}$. Dessa forma, $\forall x, y\in I$, de modo que $|x - y| < \delta$, vale que
    \begin{equation*}
        \begin{split}
            |F(x) - F(y)| & \leq L|x - y| \\
            & < L\delta \\
            & = L\cdot \dfrac{\varepsilon}{L} \\
            & = \varepsilon.
        \end{split}
    \end{equation*}
    
    Logo, $F$ é contínua em $I$.
\end{proof}

\begin{lemma}
    \label{lipschitz-continua}
    Se $A$ é uma função continuamente diferenciável e sua derivada é limitada em um intervalo, então $A$ é Lipschitz-contínua no intervalo.
\end{lemma}

\begin{proof}
    Se $A$ possui derivada limitada no intervalo $I$, existe algum $L$ de forma que $|A'(x)|\leq L, \forall x\in I$. Dessa forma, segue pelo Teorema Fundamental do Cálculo, que
    \begin{equation*}
        \begin{split}
            \left|A(x) - A(y)\right| & \leq \left|\int_x^y A'(t) ~dt\right| \\
            & \leq \int_x^y \left|A'(t)\right| ~dt \\
            & \leq L \left|x - y\right|.
        \end{split}
    \end{equation*}
    
    Com resultado seguindo pela definição de Lipschitz-continuidade.
\end{proof}

Feito isso, podemos relembrar o Teorema de Existência e Unicidade de EDO's. Para tanto iremos usar o enunciado de \cite{cruz}.

\begin{theorem}(Existência e Unicidade de Soluções de EDO's.)
    \label{existencia_e_unicidade}
    Considere o problema de valor inicial
    \begin{equation*}
        \begin{split}
            x(0) & = x_0 \\
            x'(t) & = F(x(t)).
        \end{split}
    \end{equation*}
    
    Se $F$ é Lipschitz-contínua em $[0, T]$ então existe uma solução única, $x(t)$ em $C^1(-\infty, \infty)$ com $x(0) = x_0$.
\end{theorem}

\begin{proof}
    Veja o Teorema 5.2 de \cite{cruz}.
\end{proof}

\begin{definition}(Movimento Rígido Positivo.)
    Um movimento rígido positivo é um movimento rígido que preserva a orientação.
\end{definition}

As demais definições e proposições que serão utilizadas já foram trabalhadas em aula, dessa forma optamos por omitir as mesmas. Entretanto, vale lembrar que, assim como \cite{lima}, aqui uma aplicação será dita diferenciável se a mesma for de classe $C^\infty$.

\section{Teorema Fundamental das Curvas Espaciais}
\justify

Tendo como base o ferramental apresentado na Seção \ref{preliminares}, podemos enunciar, e provar, o Teorema Fundamental das Curvas Espaciais.

\begin{theorem}(Teorema Fundamental das Curvas Espaciais.)
    Sejam $\kappa_0, \tau_0 : I\subseteq \mathbb{R}\to \mathbb{R}$ funções diferenciáveis, com $\kappa_0(t) > 0$. Então existe uma curva $\alpha : I\to \mathbb{R}^3$, parametrizada por comprimento de arco, de modo que $\kappa_\alpha(t) = \kappa_0(t)$ e $\tau_\alpha(t) = \tau_0(t)$.
    
    Além disso, se existe uma outra curva $\beta : I\to \mathbb{R}^3$ de modo que, para todo $t\in I$, seja válido que $\kappa_\beta(t) = \kappa_0(t)$ e $\tau_\beta(t) = \tau_0(t)$, então existe um movimento rígido positivo, $M : \mathbb{R}^3\to \mathbb{R}^3$, tal que $\alpha(t) = M(\beta(t))$.
\end{theorem}

\begin{proof}
    Sejam $T$, $N$ e $B$ os vetores tangente, normal e binormal, respectivamente. Sabemos que os mesmos estão em $\mathbb{R}^3$ e formam o triedro de Frenet, dessa forma, temos que eles respeitam as equações de Frenet-Serret:
    \begin{equation}
        \begin{split}
            T'(t) & = \kappa_0(t) N(t), \\
            N'(t) & = -\kappa_0(t) T(t) + \tau_0(t) B(t), \\
            B'(t) & = -\tau_0(t) N(t).
        \end{split}
        \label{frenet}
    \end{equation}
    
    Note que os vetores $T'(t)$, $N'(t)$ e $B'(t)$ também pertencem ao $\mathbb{R}^3$. Tomando $\gamma : I\to \mathbb{R}^3\times \mathbb{R}^3\times \mathbb{R}^3$ uma curva de modo que
    \[\gamma(t) = \left(T(t), N(t), B(t)\right)\]
    
    \noindent e $F : \mathbb{R}^3\times \mathbb{R}^3\times \mathbb{R}^3\to \mathbb{R}^3\times \mathbb{R}^3\times \mathbb{R}^3$ de tal que
    \[F(\gamma(t)) = \left(\kappa_0(t) N(t), -\kappa_0(t) T(t) + \tau_0(t) B(t), -\tau_0(t) N(t)\right),\]
    
    \noindent temos que $F$ é continuamente diferenciável, logo, pelo Lema \ref{lipschitz-continua}, é Lipschitz-contínua. Agora podemos fixar $t_0\in I$ e considerar a condição inicial de modo que $\gamma(t_0) = \left(e_1, e_2, e_3\right)$, onde $e_i$ são vetores da base canônica. Dessa forma, o Teorema \ref{existencia_e_unicidade} nos garante que existe uma única solução para $\gamma(t)$ de modo que $\gamma'(t) = F(\gamma(t))$.
    
    De fato, podemos definir
    \[\alpha(t) = \int_{t_0}^{t} T(u) ~du \in \mathbb{R}^3,\]
    
    \noindent de onde sai que $\alpha'(t) = T(t)$ e que $\alpha(t_0) = (0, 0, 0)$. 
    
    Isso nos garante a existência da curva, resta mostrar que a mesma está parametrizada por comprimento de arco e que qualquer outra curva que possua as mesmas funções curvatura e torção podem ser obtidas por meio de um movimento rígido. Para tanto, precisamos mostrar que os vetores $T(t)$, $N(t)$ e $B(t)$ formam uma base ortonormal positiva, isto é, que as seguintes propriedades valem
    \begin{equation}
        \begin{split}
            \langle T(t), T(t)\rangle & = 1 \\
            \langle T(t), N(t)\rangle & = 0 \\
            \langle T(t), B(t)\rangle & = 0 \\
            \langle N(t), N(t)\rangle & = 1 \\
            \langle N(t), B(t)\rangle & = 0 \\
            \langle B(t), B(t)\rangle & = 1
        \end{split}
        \label{condicoes_escalares}
    \end{equation}
    
    \noindent de onde surge o seguinte sistema de EDO's:
    \begin{equation}
        \begin{split}
            \langle T(t), T(t)\rangle' & = 2\kappa(t) \langle T(t), N(t)\rangle \\
            \langle T(t), N(t)\rangle' & = \kappa(t) \langle N(t), N(t)\rangle - \kappa(t) \langle T(t), T(t)\rangle + \tau(t) \langle T(t), B(t)\rangle \\
            \langle T(t), B(t)\rangle' & = \kappa(t) \langle N(t), B(t)\rangle - \tau(t) \langle T(t), N(t)\rangle \\
            \langle N(t), N(t)\rangle' & = - 2\kappa(t) \langle T(t), N(t)\rangle + 2\tau(t) \langle N(t), B(t)\rangle \\
            \langle N(t), B(t)\rangle' & = \tau(t) \langle B(t), B(t)\rangle - \kappa(t) \langle T(t), B(t)\rangle - \tau(t) \langle N(t), N(t)\rangle \\
            \langle B(t), B(t)\rangle' & = -2\tau(t) \langle N(t), B(t)\rangle.
        \end{split}
        \label{sistema_escalares}
    \end{equation}
    
    Note que o vetor $\lambda\left(\langle T(t), T(t)\rangle, \langle T(t), N(t)\rangle, \langle T(t), B(t)\rangle, \langle N(t), N(t)\rangle, \langle N(t), B(t)\rangle, \langle B(t), B(t)\rangle\right)$ é uma solução para \ref{sistema_escalares}. Além disso, o vetor $\tilde{\lambda} = (1, 0, 0, 1, 0, 1)$ também é. Entretanto, a mesma condição inicial que utilizamos no sistema $\gamma'(t) = F(\gamma(t))$ nos diz que $T(t_0) = e_1$, $N(t_0) = e_2$ e $B(t_0) = e_3$, ou seja, $\lambda(t_0) = \tilde{\lambda}(t_0)$. Mas, pelo teorema de unicidade da solução de uma EDO, isso implica que as duas soluções precisam ser iguais, logo, as equações em \ref{condicoes_escalares} são válidas, o que implica que $\{T(t), N(t), B(t)\}$ formam uma base ortonormada em cada $t$. Por continuidade vale que a base será positiva também, uma vez que o determinante dessa base só pode assumir os valores de $1$ e $-1$, mas, como em $t_0$ ele é igual a $1$, ele não poderá assumir o valor de $-1$.
    
    Agora, perceba que $\alpha'(t) = T(t)$, logo $\alpha$ está parametrizada por comprimento de arco. Assim, podemos escrever
    \[\kappa_\alpha(t) = \|\alpha''(t)\| = \|T'(t)\| = \|\kappa_0(t)N(t)\| = \kappa_0(t).\]
    
    Já para a torção, podemos escrever 
    \begin{equation*}
        \begin{split}
            \tau_\alpha(t) & = \dfrac{\langle \alpha'(t)\times \alpha''(t), \alpha'''(t)\rangle}{\|\alpha'(t)\times \alpha''(t)\|^2} \\
            & = \dfrac{\langle T(t)\times T'(t), T''(t)\rangle}{\|T(t)\times T'(t)\|^2} \\
            & = \dfrac{\langle T(t)\times \kappa_0(t)N(t), \kappa_0(t)N'(t)\rangle}{\|T(t)\times \kappa_0(t)N(t)\|^2} \\
            & = \dfrac{\langle \kappa_0(t)B(t), -(\kappa_0(t))^2T(t) + \kappa_0(t)\tau_0(t)B(t)\rangle}{\|\kappa_0(t)B(t)\|^2} \\
            & = \dfrac{\langle \kappa_0(t)B(t), -\left(\kappa_0(t)\right)^2T(t)\rangle + \langle \kappa_0(t)B(t), \kappa_0(t)\tau_0(t)B(t)\rangle}{\left(\kappa_0(t)\right)^2} \\
            & = \dfrac{\left(\kappa_0(t)\right)^2\tau_0(t)}{\left(\kappa_0(t)\right)^2} \\
            & = \tau_0(t).
        \end{split}
    \end{equation*}
    
    O que demonstra que $\kappa_\alpha(t) = \kappa_0(t)$ e que $\tau_\alpha(t) = \tau_0(t)$.
    
    Por fim, suponha que exista uma outra curva $\beta : I\to \mathbb{R}^3$ de modo que $\kappa_\beta(t) = \kappa_0(t)$ e $\tau_\beta(t) = \tau_0(t)$. Sabemos que $\alpha(t_0) = (0, 0, 0)$ e que o Triedro de Frenet nesse ponto são os vetores canônicos. Dessa forma, defina o movimento rígido positivo $M : \mathbb{R}^3\to \mathbb{R}^3$, de modo que $M(v) = R(v) + P(v)$, com $R : \mathbb{R}^3\to \mathbb{R}^3$ sendo definida como uma rotação de modo que
    \begin{equation*}
        \begin{split}
            T_\beta(t_0) & = e_1 & = T_\alpha(t_0) \\
            N_\beta(t_0) & = e_1 & = N_\alpha(t_0) \\
            B_\beta(t_0) & = e_1 & = B_\alpha(t_0),
        \end{split}
    \end{equation*}
    
    \noindent e $P : \mathbb{R}^3\to \mathbb{R}^3$, sendo definida como $P(v) = v - \beta(t_0)$, ou seja, $P$ será dada de modo que $P(\beta(t_0)) = \beta(t_0) - \beta(t_0) = (0, 0, 0)$, o que faz com que a curva $\beta$ respeite as condições iniciais dos nossos sistemas de EDO's, logo, novamente pela unicidade das soluções, vale que $\alpha = M(\beta)$.
\end{proof}








% \newpage
\begin{thebibliography}{9}

\bibitem{cruz} Cruz, J. (2017). The Fundamental Theorem of Space Curves. doi: \url{http://math.uchicago.edu/~may/REU2017/REUPapers/Cruz.pdf}.

\bibitem{brockveld} Brockveld, L. de L. (2018). Um estudo sobre curvas no plano e no espaço. UFSC. doi: \url{https://repositorio.ufsc.br/bitstream/handle/123456789/188642/tcc_Leonardo.pdf}.

\bibitem{terng} Terng, CL. (2005). Math 162A Lecture notes on Curves and Surfaces, Part I. doi: \url{https://www.math.uci.edu/~cterng/162A_Lecture_Notes.pdf}.

\bibitem{picado} Picado, J. (2006). Notas de aula de Geometria Diferencial.

\bibitem{coda} Codá, F. (2015). Geometria Diferencial. IMPA. \url{https://www.youtube.com/watch?v=bZiAkM6ab08}.

\bibitem{lima} Lima, R. F. de. (2016). Introdução à Geometria Diferencial. SBM.

\end{thebibliography}

\end{document}
